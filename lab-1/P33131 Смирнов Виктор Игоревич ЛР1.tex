\documentclass{article}

\usepackage[utf8]{inputenc}
\usepackage[russian]{babel}
\usepackage[a4paper, margin=1in]{geometry}
\usepackage{graphicx}
\usepackage{amsmath}
\usepackage{wrapfig}
\usepackage{multirow}
\usepackage{mathtools}
\usepackage{pgfplots}
\usepackage{pgfplotstable}
\usepackage{setspace}
\usepackage{changepage}
\usepackage{caption}
\usepackage{csquotes}
\usepackage{hyperref}
\usepackage{listings}

\pgfplotsset{compat=1.18}
\hypersetup{
  colorlinks = true,
  linkcolor  = blue,
  filecolor  = magenta,      
  urlcolor   = darkgray,
  pdftitle   = ddb-report-1-smirnov-victor-p33131,
}

\definecolor{codegreen}{rgb}{0,0.6,0}
\definecolor{codegray}{rgb}{0.5,0.5,0.5}
\definecolor{codepurple}{rgb}{0.58,0,0.82}
\definecolor{backcolour}{rgb}{0.99,0.99,0.99}

\lstdefinestyle{codestyle}{
  backgroundcolor=\color{backcolour},   
  commentstyle=\color{codegreen},
  keywordstyle=\color{magenta},
  numberstyle=\tiny\color{codegray},
  stringstyle=\color{codepurple},
  basicstyle=\ttfamily\footnotesize,
  breakatwhitespace=false,         
  breaklines=true,                 
  captionpos=b,                    
  keepspaces=true,                 
  numbers=left,                    
  numbersep=5pt,                  
  showspaces=false,                
  showstringspaces=false,
  showtabs=false,                  
  tabsize=2
}

\lstset{style=codestyle}

\begin{document}

\begin{titlepage}
    \begin{center}
        \begin{spacing}{1.4}
            \large{Университет ИТМО} \\
            \large{Факультет программной инженерии и компьютерной техники} \\
        \end{spacing}
        \vfill
        \textbf{
            \huge{Распределённые системы хранения данных.} \\
            \huge{Лабораторная работа №1.} \\
        }
    \end{center}
    \vfill
    \begin{center}
        \begin{tabular}{r l}
            Группа:        & P33131                      \\
            Студент:       & Смирнов Виктор Игоревич     \\
            Преподаватель: & Афанасьев Дмитрий Борисович \\
            Вариант:       & 776                         \\
        \end{tabular}
    \end{center}
    \vfill
    \begin{center}
        \begin{large}
            2024
        \end{large}
    \end{center}
\end{titlepage}

\section*{Ключевые слова}

База данных, PostgreSQL, системный каталог.

\tableofcontents

\section{Цель работы}

Научиться проектировать базы данных, составлять инфологические и даталогические модели данных, реализовывать их в БД PostgreSQL, научиться выполнять запросы.

\section{Текст задания}

Используя сведения из системных каталогов получить информацию о любой таблице: Номер по порядку, Имя столбца, Атрибуты (в атрибуты столбца включить тип данных, ограничение типа CHECK).

Пример вывода:
\begin{verbatim}
Таблица: Н_ЛЮДИ
No.  Имя столбца         Атрибуты
--- ------------ ------------------------------------------------------
1   ИД            Type    : NUMBER(9) NOT NULL
                  Comment : 'Уникальный номер человека'
2   ФАМИЛИЯ       Type    : VARCHAR2(25) NOT NULL
                  Comment : 'Фамилия человека'
3   ИМЯ           Type    : VARCHAR2(2000) NOT NULL
                  Comment : 'Имя человека'
4   ОТЧЕСТВО      Type    : VARCHAR2(20)  
                  Comment : 'Отчество человека'
5   ДАТА_РОЖДЕНИЯ Type    : DATE NOT NULL
                  Comment : 'Дата рождения человека'
6   ПОЛ           Type    : CHAR(1) NOT NULL
                  Constr  : "AVCON_378561_ПОЛ_000" CHECK (ПОЛ IN ('М', 'Ж'))
                  Constr  : "AVCON_388176_ПОЛ_000" CHECK (ПОЛ IN ('М', 'Ж'))
                  Comment : 'Пол человека'
7   ИНОСТРАН      Type    : VARCHAR2(3) NOT NULL
8   КТО_СОЗДАЛ    Type    : VARCHAR2(40) NOT NULL
9   КОГДА_СОЗДАЛ  Type    : DATE NOT NULL
10  КТО_ИЗМЕНИЛ   Type    : VARCHAR2(40) NOT NULL
11  КОГДА_ИЗМЕНИ  Type    : DATE NOT NULL
12  ДАТА_СМЕРТИ   Type    : DATE  
                  Comment : 'Дата смерти человека'
13  ПИН           Type    : VARCHAR2(20)  
14  ИНН           Type    : VARCHAR2(20)  
\end{verbatim}

Далее был написан SQL скрипт, создающий таблицу, аналогичную той, что в примере.

\lstinputlisting[language=SQL]{initialize.sql}

\section{Реализация скрипта}

\lstinputlisting[language=SQL]{meta.sql}
\lstinputlisting[language=SQL]{meta_display.sql}
\lstinputlisting[language=SQL]{main.sql}

\section{Таблица}

\lstinputlisting[language=bash]{out.txt}

\section{Вывод}

Данная лабораторная работа помогла мне изучить системный каталог PostgreSQL.

\begin{thebibliography}{9}

\end{thebibliography}

\end{document}